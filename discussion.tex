\paragraph{The problem of continuity and unfamiliarity with USG procedures} This problem is mostly addressed by the proposal, since all elected members must be previous USG members and in case of the committee usually committee members and since they are present at the parliament for some time at the end of the semester in which they got elected without yet being in office. 

\paragraph{The right of the committees to have a say in the choice of their chair}
This problem is addressed by the proposal, unless no committee member got elected or approved by the parliament (this should be rare). In this case, there is no good solution which still allows the student body to elect its leaders, so this is a necessary compromise. 

\paragraph{Equal representation of committees in the parliament}
This problem is not very well addressed, although the situation is somewhat improved by the committee's additional influence in the election of their (co)-chair. Furthermore, by this procedure a stronger connection and cooperation between the parliament and the committees is facilitated. 

\paragraph{Inflexibility of the election system}
This, is partially fixed, in fact the proposed election system gives the USG a lot of flexibility. 

\paragraph{Agreeing on a chair} The main weakness of this proposal is the difficulty of selecting a chair for the committees, specifically in case of a conflict between the parliament and the committee. However, assuming a minimal degree of cooperation and acceptance for compromises on both sides, such issues can still be resolvable. 